Geometrisk optik är en del av optiken där strålar används för att beskriva ljusets riktning då man beskriver ljuset som linjer som utgår från en ljuskälla och har riktningen i vilken den elektromagnetiska energin flödar\cite{MOLESINI2005257}. 

Den geometriska optiken utvecklades för sig själv men kan relateras till Maxwells ekvationer genom att anta att den allmänna lösningen till Helmholtz ekvationen inte är betydande beroende av våglängden $\lambda$ som man låter gå mot noll $\lambda\xrightarrow{} 0$. Därav fås den eikonala ekvationen som relaterar geometrisk optik till Maxwells ekvationer\cite{MOLESINI2005257}. 


Denna intuitiva idé om att ljuset är strålar som färdas i riktningen av energin gör den geometriska optiken till ett enkelt verktyg för att konstruera optiska system\cite{WINSTON20057}. Då man betraktar ljuset som strålar tillsammans med ytor som reflekterar och transmitterar strålar kan enkelt optiska system konstrueras då strålar som reflekteras och strålar som transmitteras följer reflektionslagen respektive Snells lag. 


Objekt som avger ljusstrålar om sig betraktas som källor och punkterna på objektet kan ses som punktkällor.
För att få strålarna att konvergera mot en fokalpunkt kan strålarna låtas gå mot/genom ytor som genom reflektion eller refraktion gör att strålarna konvergerar\cite{MOLESINI2005257}.
Dessa ytor är generellt linser vars konfiguration beror på den önskade förändringen av vågfronten. Linsen påverkar riktningen av strålarna, eftersom då strålarna träffar linsen och transmitteras kommer de att ändra riktning beroende på linsens utformning och då strålarna träffar linsens andra yta bryts dem ytterligare enligt Snells lag på grund av transmission\cite{Hecht}. Linsen kan bilda en bild av ett objekt genom att konvergera strålar av de många divergerande vågfronter som har bildats då ljus reflekterats mot ett objekte\cite{Hecht}, figur \ref{fig:bild_skapande} visar hur en verklig bild av ett objekt kan bildas med en positiv lins.
\begin{figure}[H]
    \centering
    \includegraphics[width=12cm]{images/bild_skapas_genom_lins.pdf}
    \caption{Bildskapande av en positiv lins.}
    \label{fig:bild_skapande}
\end{figure}
Bilden som en lins ger upphov till kan antingen vara verklig eller virtuell. Bilden anses verklig om den kan projiceras på en skärm och virtuell då den visas som att den utgår från icke existerande divergerande strålar på grund av reflektion eller refraktionseffekter\cite{MOLESINI2005257}. Figur \ref{fig:virtuell bild} visar hur en positiv lins kan ge upphov till en virtuell bild.
\begin{figure}[H]
    \centering
    \includegraphics[width=12cm]{images/virtuell bild.pdf}
    \caption{Virtuell bildbildning med positiv lins.}
    \label{fig:virtuell bild}
\end{figure}

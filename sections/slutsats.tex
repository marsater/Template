Det konstateras att den geometriska optiken är ett kraftfullt verktyg för att konstruera och analysera optiska system.
Till exempel kan Gauss linsformel användas till att relatera avstånd med fokallängd och eftersom förhållandet mellan objekt och bild samt avstånden är samma så kan förstoring räknas ut. 
Resultatet tyder även på att linser lider av avbildningsfel som kromatisk och sfärisk aborration samt distorion. Sammanfattningsvis har experimenten som genomförts givit rimliga resultat som stämmer överens med teorin och det konstateras därav att den geometriska optiken kan tillämpas för att konstruera optiska system med flera linser.